\documentclass[../Main.tex]{subfiles}

\begin{document}
\section*{\crule[blue]{.3cm}{.3cm} Introduction}
This study, Observations of bloom dynamics in the Bornholm Basin from gliders,  
 is part of the course MAR440 Marine Project from idea to realization at University of Gothenburg the autumn 2021. We are five students: Alma Holmsten, Augusta Olsson, Jonas Dietrich, Julia Mari Murata, Marie Wicklander working together with this.

The study is analysing the evolution of phytoplankton through blooms in the Bornholm basin. Two different phytoplankton blooms were observed in between Sweden and Denmark from March to early April and from mid-May to early July 2021. 

The data was collected by gliders and included ADCP backscatter, PAR (light), salinity, temperature, positions, chlorophyll a, and turbidity with the aim to describe potential physical drivers behind the productivity such as for example mixed layer depth dynamics, light availability and euphotic depth, wind stress and heat fluxes. 

Gliders are autonomous underwater vehicles able to remain at sea for long periods of time, continuously collecting data from a range of disciplines at high resolution. It has its advantages with lower costs and accuracy compared to xxxx.

The project in Bornholm was funded by the Voice of the Ocean Foundation (VOTO) in the infrastructure project Smart Autonomous Monitoring of the Baltic Sea (SAMBA). It is combining short term science objectives and the long-term vision of establishing several persistent ocean observatories across the Baltic Sea for international collaboration.
To understand the phytoplankton’s roll in the water column structure we must first understand what phytoplankton are and do. Phytoplankton is the name given to the grouping of cyanobacteria and microalgae. Cyanobacteria are single-celled and internally simple blue-green algae, and they make up the smallest, most ancient type of algae. Microalgae are also single-celled but have internal complexity. Phytoplankton play various important roles in structuring the water column. All phytoplankton photosynthesize, in fact they produce more than 50\% of the worlds oxygen (NOAA). We can calculate the concentration of phytoplankton in the water by using the fluorescence intensity which is given off by the chlorophyll-a in their system, which is located in micro-agal cells.

In our analysis we have focused on comparing blooms during two periods of time and what the driving forces are for the difference (and yyy). We will describe it further in this report. 
\end{document}
