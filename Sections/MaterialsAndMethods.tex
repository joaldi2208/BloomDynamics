\documentclass[../Main.tex]{subfiles}

\begin{document}
\section*{\crule[blue]{.3cm}{.3cm} Materials and methods}
\subsection*{\crule[blue]{.2cm}{.2cm} Glider setup}
Two SeaExplorer gliders manufactured by Alseamar (France) completed a transect of \SI{40}{km} from \SI{55.57}{\degree}N and \SI{16.43}{\degree}E to \SI{55.25}{\degree}N and \SI{15.98}{\degree}E between March 5, 2021, and October 5, 2021. 
Each glider performed around 1500 dives up to a maximum depth of \SI{80}{m} per mission. 
Every dive took 20min and covered a horizontal distance of \SIrange{300}{400}{m}. 

\\ 
Gliders were equipped with sensors measuring \emph{conductivity-temperature-depth} (CTD) (RBR Legato sensor), fluorescence (Wetlabs ECO Puck FLBBCD - EXP sensor), and \emph{photosynthetically active radiation} (PAR) (Seabird OCR504i sensor). 
The Wetlabs sensor emits light at \SI{470}{nm} wavelength and detects fluorescence emission of chlorophyll at \SI{695}{nm}. 
The Seabird OCR504i sensor measures PAR in the range between 400 to \SI{700}{nm}\supercite{ACSA2014}.

\subsection*{\crule[blue]{.2cm}{.2cm} Data}
Preprocessed glider data was made available by the supervisors of this project. Preprocessing comprised the correction of the thermal lag error in the CTD data.
Discrepancies in time and space between temperature and conductivity measurements result from the sensor arrangement\supercite{Garau2011}.

\\
Wind speed data comes from the Climate Data Store\supercite{era}. 
To estimate wind stress, we selected the $u$- and the $v$-component of wind at \SI{10}{m} above the surface. 
The $u$-component represents the east wind direction and the $v$-component the north wind direction\supercite{era2}. 
We further limited the data set to the spatial section between \SI{55.58}{\degree}N, \SI{16.44}{\degree}E, \SI{55.25}{\degree}N and \SI{15.90}{\degree}E, and the period between March 2021 to October 2021. 

\subsection*{\crule[blue]{.2cm}{.2cm} Analysis}
We analyzed the data in Python.
To improve the visual interpretability, we log-transformed the data of chlorophyll and PAR.

\\
Using the GSW package, we calculated the potential density from salinity and temperature, and $N^2$ by additionally considering pressure\supercite{gsw}. 
Total wind speed at \SI{10}{m} height above the surface $U_{10m}$ can be evaluated as $U_{10m}$ = $\sqrt[]{u^2 + v^2}$. 
Following Mehrfar et al. (2018), we calculated the drag coefficient with $C_{Drag} = 0.001\cdot1.2\cdot  (1.1+0.035\cdot U_{10m})$. 
Wind stress $\tau$ acts as a proxy for turbulent energy exchange between water and air.
Mathematically, it can be approximated by the product of air density $\rho_{Air}$, the total wind speed at \SI{10}{m} above the surface, and the drag coefficient with $\tau = \rho_{Air}\cdot C_{Drag}\cdot U_{10m}^2$\supercite{Mehrfar2018}.

\\
We estimated the EZD at pressure values, where PAR equates to \SI{1}{\%} of the surface value\supercite{Lee2007}.
To assess MLD, we calculated the mean density above \SI{4}{m} and defined $\SI{0.125}{kg/m^3}$ as threshold difference to the surface. 
We assumed the MLD to occur at pressure values, where mean density values exceed the threshold. 
\\
\\
Data and further information are available on \href{https://github.com/joaldi2208/BloomDynamics}{GitHub}.

\end{document}
