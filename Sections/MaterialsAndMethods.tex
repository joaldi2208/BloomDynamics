\documentclass[../Main.tex]{subfiles}
\usepackage{hyperref}
\hypersetup{
    colorlinks=true,
    urlcolor=magenta
    }

\begin{document}
\section{Materials and methods}
\subsection{Glider setup}
Two calibrated SeaExplorer sea gliders manufactured by Alseamar (France) accomplished a transect of 40km from 55.570927°N and 16.432385°E (B1) to 55.250000°N and 15.983300°E (B2) between March 5, 2021 and October 5, 2021. 
Each glider completed around 1500 dives up to a max depth of 80 m per mission with one mission taking 20 min and covering a total distance of 300 to 400 m. 
Mission numbers of the gliders are 18, 19, 20, 21, 24 (Mission SEA055) and 38, 39, 40, 41, 42, 43 (Mission SEA061). 
\\ 
Gliders were equipped with sensors measuring temperature and salinity (RBR Legato sensor), fluorescence and optical backscattering (Wetlabs ECO Puck FLBBCD - EXP sensor), dissolved oxygen (Rinko AROD-FT sensor), and photosynthetically active radiation (PAR) (Seabird OCR504i sensor). 
The Wetlabs sensor is an active sensor emitting light at 470 nm wavelength and detecting fluorescence emission of chlorophyll at 695 nm, and phycocyanin backscattering at 650 nm. 
It thus provides an estimation of phytoplankton concentration. 
The Seabird OCR504i sensor measures PAR in the range between 400 to 700 nm \cite{Alseamar2020}.

\subsection{Data and analysis}
Glider data used in this report stems from one of the gliders. 
Preprocessed data was made available by the supervisors of this project. 
Preprocessing contained despiking, correcting the thermal lag on the CTD, chlorophyll quenching, vertical gridding, and daily binning. 
\\
We analyzed the data in Python. We calculated the potential density from absolute salinity and conservative temperature assuming a pressure of 0 dbar as reference \cite{GSW2017}.
\\ 
Wind speed data comes from the Climate Data Store \cite{Era5}. 
We selected the u- and the v-components of wind at 10 m above surface as input variables and limited the data to the spatial section between 55.58°N, 16.44°E, 55.25°N and 15.98°E, and the time period between March 2021 to October 2021. 
The u-component represents the horizontal directional wind and the v-component represents the vertical directional wind. 
The sum of both wind vectors represents the magnitude of total wind speed at 10 m height above the surface. Based on that, the drag coefficient can be calculated. 
Wind stress indicates the extent of energy exchange between water and air.
Quantitatively, it can be approximated by the product of air density, total wind speed at 10 m above the surface and the drag coefficient \cite{Mehrfar2018}. Detailed mathematical operations can be seen in Appendix 1.
\\
Data and further information can be obtained on \href{https://github.com/joaldi2208/BloomDynamics}{GitHub}.
\\
\\
Do we want to put the formulas in the Appendix? 
\\
Calculate drag coefficient $C_D$:
\\
\\
$C_D = 0.001(1.1+0.035*U_10^2)$  
\\
\\
Calculate wind stress from air density $\rho_a$, total wind speed at 10 m above surface $U_10$ drag coefficient $C_D$:
\\
\\
$\tau = \rho_a*C_D*U_10^2$ 
\end{document}

