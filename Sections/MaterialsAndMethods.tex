\documentclass[../Main.tex]{subfiles}
\usepackage{hyperref}
\hypersetup{
    colorlinks=true,
    urlcolor=magenta
    }

\begin{document}
\section{Materials and methods}
\subsection{Glider setup}
Two calibrated SeaExplorer sea gliders manufactured by Alseamar (France) accomplished a transect of 40km from 55.570927°N and 16.432385°E (B1) to 55.250000°N and 15.983300°E (B2) between March 5, 2021 and October 5, 2021.
Each glider completed around 1500 dives up to a max depth of 80 m per mission with one mission taking 20 min and covering a total distance of 300 to 400 m.
Mission numbers of the gliders are 18, 19, 20, 21, 24 (Mission SEA055) and 38, 39, 40, 41, 42, 43 (Mission SEA061).
\\
Gliders were equipped with sensors measuring temperature and salinity (RBR Legato sensor), fluorescence and optical backscattering (Wetlabs ECO Puck FLBBCD - EXP sensor), dissolved oxygen (Rinko AROD-FT sensor), and photosynthetically active radiation (PAR) (Seabird OCR504i sensor).
The Wetlabs sensor is an active sensor emitting light at 470 nm wavelength and detecting fluorescence emission of chlorophyll at 695 nm, and phycocyanin backscattering at 650 nm.
It thus provides an estimation of phytoplankton concentration.
The Seabird OCR504i sensor measures PAR in the range of 400 to 700 nm \cite{x}.

\subsection{Data and analysis}
Glider data used in this report comes from one glider.
Preprocessed data was made available by the supervisors of this project.
Preprocessing contained despiking, correcting the thermal lag on the CTD, chlorophyll quenching, vertical gridding, and daily binning.
\\
Data for ocean surface stress comes from the Climate Data Store \cite{x}.
We limited the data to the spatial section between 55.58°N, 16.44°E, 55.25°N and 15.98°E, and the time period between March 2021 to October 2021.
\\
We analyzed the data in Python.
Data and further information can be obtained \href{https://github.com/joaldi2208/BloomDynamics}{here}.

\end{document}
