\documentclass[../Main.tex]{subfiles}

\begin{document}
\section*{\crule[blue]{.3cm}{.3cm} Discussion}
In this study, high-resolution data on both seasonal and vertical distributions of physical factors were examined to find out what influenced phytoplankton productivity. The results showed two different phytoplankton blooms from March to early April and from mid-May to early July (Fig. \ref{fig:all}D). One physical factor affecting the formation of phytoplankton blooms is the depth of the mixed layer, which corresponds to the position of the pycnocline.  
According to Sverdrup's theories, phytoplankton blooms can form if the depth of the mixed water layer is less than the critical depth\supercite{Sverdrup1953OnCF}. The critical depth is the depth at which solar radiation can just balance respiration. If the depth of the mixed layer were to expand deeper than the critical depth, the solar intensity would be too low to sustain a community of phytoplankton. For the two observed blooms, the critical depth varied because the data had seasonal variations with different solar irradiance. 
However since the blooms occurred both had a mixed layer that did not expand deeper than the critical depth. Another term by Sverdrup is the compensation depth, which has been called the productive layer, which is the depth at which the rate of production of organic matter by photosynthesis is equivalent to the rate of decomposition of organic matter by plant respiration\supercite{Sverdrup1953OnCF}. 
The depth below the compensation depth would therefore have no net production. 

Also, both light and nutrients are required for a phytoplankton bloom to form\supercite{munkes2021cyanobacteria}. 
As the study area is located in the Baltic Sea, which is known to be eutrophic, nutrient deficiencies are unlikely\supercite{munkes2021cyanobacteria}. 
For the second bloom, in terms of light, the highest light irradiance corresponds to the highest chlorophyll value (Fig. \ref{fig:all}D and \ref{fig:all}E). However, there is less light available during the first bloom, but enough to form this bloom. Intense solar radiation thus leads to a reduction in phytoplankton growth due to photoinhibition\supercite{edwards2016phytoplankton}. 
This may be the reason why no high chlorophyll values are observed at the surface during the second bloom, whereas high chlorophyll values were present at the surface during the first bloom. 

As for the salinity gradient, it is more homogeneous in spring, with a pycnocline starting at \SI{50}{m} depth. This pycnocline is also observed in the summer months. However, during the summer months, another shallower pycnocline is observed at \SI{18}{m} and \SI{25}{m} depth (Fig. \ref{fig:all}C). This pycnocline corresponds to the temperature profile, where a thermocline also occurs in summer from late July to October at the same depth (Fig. \ref{fig:all}C and \ref{fig:all}B). This stratification is probably formed due to lower wind stress and increased water temperature, with the lower wind stress contributing to the establishment of a warmer surface layer\supercite{carey2012eco}.

In addition, this stratification at shallower depths during the summer months contributes to less vertical water circulation, trapping phytoplankton at shallower depths with more ideal solar irradiance, leading to a greater likelihood of a bloom taking shape. This affects the position of phytoplankton as they have limited ability to move through the water mass\supercite{kase2018phytoplankton}.
Since chlorophyll concentration penetrates deeper during spring (March-May) which has higher wind stress and no shallow stratification, while during summer (June-October) there is less wind stress and shallower stratification, this results in no chlorophyll at deeper depths for the second bloom (Fig. \ref{fig:all}D).

However, no phytoplankton bloom has been observed after July, although there seems to be a stratification and solar radiation favourable for a bloom. One explanation for this is that the nutrients have been consumed by the previous phytoplankton bloom and therefore no new bloom occurs after July. Since there is stratification, this may contribute to less vertical exchange of new nutrient-rich water, which would result in less nutrients. As no nutrient data were observed, this can only be speculation without certainty. 

The two blooms are influenced by different physical factors as there are seasonal variations. There is no clear correlation between bloom and water column stratification of density, temperature or salinity for the first bloom. In contrast, the second bloom appears to be formed primarily due to stratification that occurs due to lower wind stress and higher temperature. 

\end{document}
