\documentclass[../Main.tex]{subfiles}

\begin{document}
\section*{\crule[blue]{.3cm}{.3cm} Discussion}
In this study, high-resolution data on both seasonal and vertical distributions of physical factors were examined to find out what influenced phytoplankton productivity. The results showed two different phytoplankton blooms from March to early April and from mid-May to early July (Fig. \ref{fig:all}D). One physical factor affecting the formation of phytoplankton blooms is the depth of the mixed layer, which corresponds to the position of the pycnocline.  
According to Sverdrup's theories, phytoplankton blooms can form if the depth of the mixed water layer is less than the critical depth, which is the depth at which solar radiation can just balance respiration.\supercite{Sverdrup1953OnCF}. If the depth of the mixed layer were to expand deeper than the critical depth, the solar intensity would be too low to sustain a community of phytoplankton. Since the blooms occurred, both had a mixed layer that did not expand deeper than the critical depth. Another term by Sverdrup is the compensation depth, also called the productive layer, which is the depth at which the rate of production of organic matter by photosynthesis is equivalent to the rate of decomposition of organic matter by plant respiration\supercite{Sverdrup1953OnCF}. 
The depth below the compensation depth would therefore have no net production. 

In addition, both light and nutrients are required for a phytoplankton bloom to form\supercite{munkes2021cyanobacteria}. The euphotic depth is the maximum depth of the light zone suitable for phytoplankton photosynthesis (7). Comparing the two blooms, the first bloom had a shallower euphotic depth, while the second bloom had a deeper euphotic depth. Due to the deeper euphotic depth during the second bloom, this results in phytoplankton being able to photosynthesize at a greater depth. Whereas in the first bloom, the phytoplankton can photosynthesize at a shallower depth  (Fig. \ref{fig:all}A). The first bloom had less light penetrating the water column, while the second bloom had higher light penetration at greater depths (Fig. \ref{fig:all}E). 
For the second bloom, in terms of light, the highest light irradiance corresponds to the highest chlorophyll value (Fig. \ref{fig:all}D and \ref{fig:all}E). 
However, no high chlorophyll values were observed at the surface during the second bloom, whereas high chlorophyll values were present at the surface during the first bloom (Fig. \ref{fig:all}D). This could be explained with the Deep Chlorophyll Maxima (DCM), known as subsurface peaks in chlorophyll concentration that may correspond to peaks in primary production and abundance of phytoplankton. The DCM formation occurs between the nutrient-poor, light-rich upper mixed layer and a nutrient-rich, light-limited bottom mixed layer coinciding with the thermocline that separates these two layers (6). The reason why no high chlorophyll concentrations were observed at the surface during the second bloom may therefore be due to DCM formation, as there is a strong thermocline corresponding to the depth at which the high chlorophyll concentration occurs. In addition, light irradiation during the second bloom would therefore penetrate deeper due to less chlorophyll at the surface, suggesting less phytoplankton, leading to less absorption of solar radiation. Whereas the first bloom had higher chlorophyll, indicating more phytoplankton, resulting in more absorption of solar radiation at the surface. 

As for the density gradient, it is more homogeneous in spring, with a pycnocline starting at \SI{50}{m} depth (Fig. \ref{fig:all}C). This pycnocline is also observed in the summer months. However, during the summer months, another shallower pycnocline is observed at \SI{18}{m} and \SI{25}{m} depth (Fig. \ref{fig:all}C). This pycnocline corresponds to the temperature profile, where a thermocline also occurs in summer from late July to October at the same depth (Fig. \ref{fig:all}C and \ref{fig:all}B). This stratification is probably formed due to lower wind stress and increased water temperature, with the lower wind stress contributing to the establishment of a warmer surface layer\supercite{carey2012eco}.

In addition, this stratification at shallower depths during the summer months, which is presented in the N2-plot (Fig. \ref{fig:all}C), contributes to less vertical water circulation, trapping phytoplankton at shallower depths with more ideal solar irradiance; this leads to a greater likelihood of a bloom taking shape. This affects the position of phytoplankton as they have limited ability to move through the water mass\supercite{kase2018phytoplankton}.
Chlorophyll concentration penetrates deeper in spring (March-May), due to higher winds, leading to turbulent mixing of the upper layer and thus less stratification. In summer (June-October), there is less wind stress and stratification is stronger, resulting in no chlorophyll at deeper depths for the second bloom (Fig. \ref{fig:all}D).

Furthermore, there seems to be a small phytoplankton bloom at the end of September, which is weaker than the first two blooms, although there seems to be stratification and solar radiation that favours a stronger bloom (Fig. \ref{fig:all}D and \ref{fig:all}E). One explanation for this weaker bloom is that the nutrients have been consumed by the previous phytoplankton bloom. Since there is stratification, this may contribute to less vertical exchange of new nutrient-rich water, which would result in less nutrients. However, during the third bloom, wind stress increases, leading to less stratification, which may contribute to a greater amount of nutrients that would be sufficient to support this bloom.  As no nutrient data was observed, this can only be speculation without certainty. 

The two blooms are influenced by different physical factors as there are seasonal variations. There is no clear correlation between bloom and water column stratification of density, temperature or salinity for the first bloom. In contrast, the second bloom appears to be formed primarily due to stratification that occurs due to lower wind stress and higher temperature.




% TWO NEW SOURCES, reffered in the text as (6) and (7).
%6. Fernand, L., Weston, K., Morris, T. et al. The contribution of the deep chlorophyll maximum to primary production in a seasonally stratified shelf sea, the North Sea. Biogeochemistry 113, 153–166 (2013). https://doi.org/10.1007/s10533-013-9831-7

%7. @article{article,
%author = {Khanna, D.R. and Bhutiani, Rakesh and Chandra, K.S.},
%year = {2009},
%title = {Effect of the Euphotic Depth and Mixing Depth on Phytoplanktonic Growth Mechanism},
%volume = {3},
%journal = {International Journal of Environmental Research}
%}


\end{document}
