\documentclass[../Main.tex]{subfiles}

\begin{document}
\section{Discussion}
Two different phytoplankton blooms were observed from March to early April and from mid-May to early July (fig chloro). Both light and nutrients are required for a phytoplankton bloom to form (1). As the study area is located in the Baltic Sea, which is known to be eutrophic, nutrient deficiencies are unlikely (1). For the second bloom, in terms of light, the highest light irradiance corresponds to the highest chlorophyll value (Fig. chloro + par). However, there is less light available during the first bloom, but enough to form this bloom. Intense solar radiation thus leads to a reduction in phytoplankton growth due to photoinhibition (4). This may be the reason why no high chlorophyll values are observed at the surface during the second bloom, whereas high chlorophyll values were present at the surface during the first bloom.  

Another physical factor affecting the formation of phytoplankton blooms is the depth of the mixed layer, which corresponds to the position of the pycnocline. As for the salinity gradient, it is more homogeneous in spring, with a pycnocline starting at 50 meter depth. This pycnocline is also observed in the summer months. However, during the summer months, another shallower pycnocline is observed at 18 and 25 meter depth (Fig. sal). This pycnocline corresponds to the temperature profile, where a thermocline also occurs in summer from late July to October at the same depth (fig. sal + temp). This stratification is probably formed due to lower wind stress and increased water temperature, with the lower wind stress contributing to the establishment of a warmer surface layer (2).

In addition, this stratification at shallower depths during the summer months contributes to less vertical water circulation, trapping phytoplankton at shallower depths with more ideal solar irradiance, leading to a greater likelihood of a bloom taking shape. This affects the position of phytoplankton as they have limited ability to move through the water mass (3). Since chlorophyll concentration penetrates deeper during spring (March-May) which has higher wind stress and no shallow stratification, while during summer (June-October) there is less wind stress and shallower stratification, this results in no chlorophyll at deeper depths for the second bloom (Fig. Chlorophyll).

However, no phytoplankton bloom has been observed after July, although there seems to be a stratification and solar radiation favourable for a bloom. One explanation for this is that the nutrients have been consumed by the previous phytoplankton bloom and therefore no new bloom occurs after July. Since there is stratification, this may contribute to less vertical exchange of new nutrient-rich water, which would result in less nutrients. As no nutrient data were observed, this can only be speculation without certainty. 

\end{document}