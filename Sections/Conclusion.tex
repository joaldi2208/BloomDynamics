\documentclass[../Main.tex]{subfiles}

\begin{document}
\section*{\crule[blue]{.3cm}{.3cm} Conclusion}
In summary, the daily glider data from March to October show two different major phytoplankton blooms and one minor bloom after Septemberwith different vertical distributions. These differences could be attributed to different physical parameters.
During the first bloom, which occurred from March to early April, relatively high wind stress and relatively low temperatures resulted in no stratification, allowing phytoplankton to penetrate deeper into water regions.
In contrast, the second bloom, which occurred from mid-May to early July, was strongly influenced by high radiation and warm water temperatures and low wind stress.
These factors resulted in highly pronounced stratification, allowing phytoplankton to stay in a similar water depth over a period of about two months.
Lack of nutrients could be a reason for the absence of a bloom in the surface layer.
The smaller bloom after September shows lower amount of phytoplankton as expected from physical drivers investigated.
Again, nutrients could be limiting for phytoplankton growth.
These different blooms demonstrate the adaptability of phytoplankton to different physical factors and their effects on the distribution of phytoplankton in the water stratification.
The result shows a high complexity of phytoplankton growth, which cannot be explained exclusively by the physical factors studied.

Since the project aimed only at a coarse triage of the data, further investigation of the dataset, even at even higher resolution, will allow the study of phytoplankton bloom dynamics at both small and large scales and may provide hidden insights that were previously undiscovered.
A major focus should be on examining shape, lifespan, and also factors that inhibit growth enhancement, such as the presence of nutrients.
Distinguishing between different phytoplankton species could also help to reduce complexity, as each species is adapted to different environmental conditions. 
\end{document}
