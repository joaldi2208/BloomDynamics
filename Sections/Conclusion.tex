\documentclass[../Main.tex]{subfiles}

\begin{document}
\section*{\crule[blue]{.3cm}{.3cm} Conclusion}
In summary, the daily glider data from March to October show two different phytoplankton blooms with different vertical distributions. These differences could be attributed to different physical parameters.
During the first bloom, which occurred from March to early April, relatively high wind stress and relatively low temperatures resulted in no stratification, allowing phytoplankton to penetrate deeper into water regions.
In contrast, the second bloom, which occurred from mid-May to early July, was strongly influenced by high radiation and warm water temperatures and low wind stress. These factors resulted in highly pronounced stratification, allowing phytoplankton to live in a similar water depth over a period of about two months. The high radiation could prevent the growth of phytoplankton in the surface layer.
These different blooms demonstrate the adaptability of phytoplankton to different physical factors and their effects on the distribution of phytoplankton in the water stratification.

Since the project aimed only at a coarse triage of the data, further investigation of the dataset, even at even higher resolution, will allow the study of phytoplankton bloom dynamics at both small and large scales and may provide hidden insights that were previously undiscovered.
A major focus should be on examining shape, lifespan, and also factors that inhibit growth enhancement, such as the presence of nutrients.
\end{document}
