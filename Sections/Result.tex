\documentclass[../Main.tex]{subfiles}

\begin{document}
\section{Result}
The density plot shows a homogenized water column down to 50 m where it then starts to show pycnoclines in the denser bottom water in early spring. This vertical stratification is stable until the end of April where the water column starts to become more stratified, showing pycnoclines closer to the surface. In August the water is fully stratified with pycnoclines showing up at ~18, ~25, ~50, ~53 and ~60 m. In beginning of September the water column is starting to lose the upper pycnocline at 18 m and in October the water is homogenized down to ~30 m. The temperature is showing a similar distribution to that of the density, with a homogenized temperature distribution down to ~50m in April where is shows a thermocline at ~50 m. The temperature then in the upper water layer and in the end of July the water is showing strong thermoclines at ~18 m, ~25m, and a smaller patch of cold water at 45 m before it gets to the same temperature as in the column of water between 25m-45m. In October the water still shows a thermocline at 18 m but with colder water in the upper layer and a thermocline starting to build up again in the deeper waters, this time with colder water in the middle layer and warmer water at the bottom. The salinity is following the trends of that of density and temperature, with haloclines starting to form in the upper layer water in end of April with patches of high salinity water coming through until the halocline is stabilized in the middle of June at ~25 m. The halocline at ~25 m becomes deeper through the summer months with a except for a maximum in beginning of september, but in the end of October the halocline is at ~30m. Chlorophyll is showing a change in concentration in the upper water layer down to ~30 m in early March beginning of April with the highest concentrations showing up at 0-15 m of depth. The chlorophyll is then homogenized in the water column between early April until end of May where it start to shown an increasing concentration again at between 0-15 m. This concentration band is increasing through that of June, with the highest concentrations at ~25m. The wind stress is of the water is lower over the during end of May until beginning of August where is starts to increase and in October the wind stress is showing the strength of that in early spring. Between beginning of May and until he beginning of September the light radiation starts to show increasing values in the upper water layer, ~1-2 m. 

\end{document}