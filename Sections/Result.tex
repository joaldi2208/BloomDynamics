\documentclass[../Main.tex]{subfiles}

\begin{document}
\section*{\crule[blue]{.3cm}{.3cm} Result}
The the glider data for density (Fig.\ref{fig:all}E) shows a homogenized water column down to about \SI{50}{\metre} where it then starts to show pycnoclines in the denser bottom water.
 This vertical stratification is stable until the end of April where the water column starts to become more stratified, showing pycnoclines closer to the surface.
In August the water is fully stratified with pycnoclines showing up at approximately \SIlist{18;25;50;53;60}{\metre}. 
In beginning of September the water column is starting to lose the upper pycnocline at $\sim$\SI{18}{\metre} and in October the water is homogenized down to $\sim$\SI{30}{\metre}.

The temperature data (Fig. \ref{fig:all}B) is showing a similar distribution to that of the density. 
In early spring, the water column has a homogenized temperature distribution down to about \SI{50}{\metre} where it then shows a thermocline. 
The temperature then starts to rise in the upper water layer and in the end of April the water column start to stratify. 
In July the water is showing strong thermoclines at about \SI{18}{\metre} and \SI{25}{\metre}, and a smaller patch of cold water at $\sim$\SI{45}{\metre} before it gets to the same temperature as in the column of water between $\sim$\SI{25}{\metre} and $\sim$\SI{45}{\metre}. 
In October the water still shows a thermocline at $\sim$\SI{18}{\metre} but with colder water in the upper layer and a thermocline starting to build up again in the deeper waters, this time with colder water in the middle layer and warmer water at the bottom. 

The salinity data (Fig. \ref{fig:all}C)is following the trends of that of density and temperature, with haloclines starting to form in the upper water layer in end of April with patches of high salinity water coming through until the halocline is stabilized in the middle of June at approximately \SI{25}{\metre}. 
The halocline at $\sim$\SI{25}{\metre} becomes deeper through the summer months with an exception for a patch of high salinity water breaking through to the surface in beginning of September, but in the end of October the halocline is down to about \SI{30}{\metre}. 

Chlorophyll data (Fig. \ref{fig:all}D) is showing a change in concentration in the upper water layer down to $\sim$\SI{30}{\metre} in March until beginning of April with the highest concentrations showing up at \SIrange{0}{15}{\metre} of depth. 
The chlorophyll concentration then declines and the water is showing a vertical homogenized profile in between early April until end of May where it start to show an increasing concentration again at between approximately \SIrange{0}{15}{\metre}. 
This chlorophyll concentration band is increasing through that of June, with the highest concentrations at about \SI{25}{\metre}.

Between beginning of May until the beginning of September light radiation (Fig. \ref{fig:all}E) starts to show increasing values in the upper water layer in a depth of approximately \SIrange{1}{2}{\metre}.

The wind stress of the water (Fig.\ref{fig:all}F) is high in early spring and start to decrease from May until beginning of August, then it starts to pick up again and in October the wind stress is showing the strength of that in early spring. 



\end{document}
