\documentclass[../Main.tex]{subfiles}

\begin{document}
\section*{\crule[blue]{.3cm}{.3cm} Result}
The temperature section (Fig. \ref{fig:all}A) shows how the water column in early spring has a homogenized temperature distribution down to about \SI{50}{\metre} where a thermocline appears. The temperature then starts to rise in the upper water layer and by the middle of May the water column start to stratify. In July the water column shows a thermocline at about \SI{18}{m}, and a smaller patch of cold water at $\sim$\SI{45}{\metre} before the water approches to the same temperature as in the column of water between $\sim$\SI{18}{\metre} and $\sim$\SI{45}{\metre}. 
In October the water column continue to show a thermocline at $\sim$\SI{18}{\metre} but with colder water in the upper water layer and a thermocline starting to build up again in the deeper waters, with colder water in the middle layer and warmer water at the bottom. 

In (Fig. \ref{fig:all}A) the MLD shows a deeper setting in early spring and in the beginning of May the MLD begin to travel higher up in the water column and from June-August the MLD appears around $\sim$\SI{18}{\metre}. In beginnning of September the MLD shows a trend of increasing in depth. 

The EZD (Fig. \ref{fig:all}A) appears in the upper water layer at about \SIrange{0}{10}{m} in early spring and increases in depth in beginning of May to around \SI{18}{m}. In early June and September the EZD reaches deeper depths of around $\sim$\SI{50}{\metre}, but returns to $\sim$\SI{18}{\metre} in-between the two occasions.

The density section (Fig.\ref{fig:all}B) shows a homogenized water column down to just below \SI{50}{m} where the denser bottom water creates two pycnoclines at approximately \SI{50}{m} and \SI{55}{m}. The vertical stratification is stable until the end of April where the water column then starts to stratify and in July the water column is fully stratified with pycnoclines showing up at approximately \SIlist{18;50;55}{\metre}. 
Through the summer the upper pycnocline increases in depth and in beginning of October the water is homogenized down to $\sim$\SI{25}{\metre}.

The salinity data (Fig. \ref{fig:all}C) shows two haloclines at around \SI{50}{m} and \SI{55}{m}, similar to the density section. Small variations can be seen in the water column above \SI{50}{m} of depth, with salinity ranging between \SIrange{7}{8}{PSU}. From beginning of May until August, a trend of less salty water in the surface water can be observed and a weak halocline appears at around \SI{25}{m}.

Chlorophyll data(Fig. \ref{fig:all}D) shows a high chlorophyll concentration in the upper water layer down to $\sim$\SI{30}{\metre} in March until the beginning of April. Small patches of high chlorophyll concentrations near the bottom in April can also be observed. 
In the middle of May until July a high chlorophyll concentration band hovers at around \SI{25}{m}. The chlorophyll concentration continues to be higher in the surface layer compared to the deep water through the summer. In the beginning of autumn there is an increase of chlorophyll concentration from the water surface down to about \SI{25}{m}.

Between beginning of April until July the PAR radiation(Fig. \ref{fig:all}E) increases in depth, with the highest irradiance in May-June.

The wind stress (Fig. \ref{fig:all}F) shows a trend of being stronger in early spring and autumn and weaker in between the months June-August.

The $N^2$ plot (Fig. \ref{fig:all}G) shows a strong stratification of the water column in the beginning of June at around  $\sim$\SI{18}{\metre}. The stratification continues through summer with the highest value in the beginning of August. The $N^2$ stratification shows a low convection ability of the water column. 

\end{document}
