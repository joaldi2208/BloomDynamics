\documentclass[../Main.tex]{subfiles}

\begin{document}

\begin{tcolorbox}[colback=light-orange, boxrule=0pt]
  \begin{multicols}{3}
    \textcolor{blue}{\textbf{ABSTRACT:}}
Phytoplankton blooms annually occur in spring and fall.
Because of their great ecological impact, phytoplankton blooms are frequently addressed in research.
Using satellite imagery and water samples, effects of physical parameters have been studied on phytoplankton growth.
Physical parameters have mostly been limited to temperature, light, and density.
The lack of precision and continuity of collected data often challenges finding correlations.
Here we show an overview of a new high-resolution dataset taken by gliders in the Baltic Sea from March to October 2021.
The data provide insight into two phytoplankton blooms that occurred from mid-March to April and from mid-May to early July.
We revealed that the blooms do not depend on the same physical drivers. 
Phytoplankton blooms are caused by different factors that largely influence the distribution of phytoplankton in water stratification. 
Our results show the possibility to find relationships between different physical drivers of phytoplankton blooms when analyzing high- resolution and continuous data. 
Hereby gliders are bleeding edge vehicles to data collection.
\ \\
\ \\
    \includegraphics[width=0.33\textwidth]{Glider.png}
 \end{multicols}
\end{tcolorbox}
\end{document}
